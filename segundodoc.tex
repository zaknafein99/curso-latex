\documentclass[a4paper,12pt]{exam}
 
\usepackage[spanish]{babel}
\usepackage[utf8]{inputenc}
\usepackage[T1]{fontenc}
\usepackage[colorlinks=true,linkcolor=blue,urlcolor=blue,citecolor= blue,anchorcolor=blue]{hyperref}
\usepackage[dvipsnames]{xcolor}

\usepackage{amsmath,amsfonts,amssymb,graphicx,multicol,tikz,color,colortbl,multirow,anysize,lmodern,
mdwlist,enumerate,textcomp,fancybox,caption}

\setlength{\headheight}{1cm}
\setlength{\textwidth}{0cm}
\marginsize{1.5cm}{1.5cm}{1.5cm}{1.5cm}

%Esta línea da el formato de la numeración, del formato por defecto (a) a utilizar solo paréntesis derecho a)
\renewcommand{\partlabel}{\thepartno)}

\header{Encabezado izq. sup. \\ Encabezado izq. inf.}{Encabezado central sup. \\ Encabezado central inf.}{Encabezado der. sup. \\ Encabezado der. inf.}
\headrule

\footer{}{Página \thepage \,de \numpages}{}
\footrule

\begin{document}

\vspace{0.5cm}

\begin{questions}
 
\question Primera pregunta
	\begin{parts}
	\part Primera parte de pregunta
		\begin{subparts}
			\subpart Primera subparte
			\subpart Segunda subparte
		\end{subparts}
	\part Segunda parte de pregunta
	\end{parts}
 
\question Segunda pregunta
	\begin{align}
		2x+4 & =0 \notag \\
		2x & =-4 \label{opuesto}\\
		x & =-2
	\end{align}
	¿Cómo se obtuvo la ecuación \ref{opuesto}?
	
\end{questions}

% Alineación izquierda de funciones
\[
\left \{
\begin{array}{l}
5x+5y=10\\
x+y=2
\end{array}
\right .
\]

% Alineación en columnas de funciones
\[
\left \{
\begin{array}{rcl}
5x & +5y & =10\\
x & +y & =2
\end{array}
\right .
\]

\[
\left(
\begin{array}{rcl}
2x^2 & +2x & =0\\
x^2 & +x & =0
\end{array}
\right)
\]

\begin{center}
\begin{tabular}{r|c|l}
5x & +5y & =10 \\
\hline
x & +y & =2
\end{tabular}

\end{center}

\end{document}
